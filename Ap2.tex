%
%
%
\chapter{Sobre o Padrão em \LaTeX{}}

\emph{Este documento foi constuído utilizando o \LaTeX{}.
Os arquivos fonte deste documento formam o próprio padrão de dissertação
do PPGEQ.}

\section{Obtendo o padrão em \LaTeX{}}

Os arquivos fonte deste documento estão livremente disponíveis e podem ser obtidos no seguinte endereço:
\href{https://github.com/rpseng/PadraoLatex}{https://github.com/rpseng/PadraoLatex}.

O usuário deverá seguir os comentários presentes nos arquivos fonte para montar
sua própria tese ou dissertação.
Este padrão \LaTeX{} automatiza da capa às referências.
Não é preciso fazer nada para que todas as citações e referências estejam corretas.
Assim como as listas de figuras, tabelas, símbolos e etc.

Alguns comentários sobre pacotes úteis são apresentados nas seções que seguem.

\section{Lista de símbolos}

O padrão \LaTeX{} do PPGEQ automatiza a criação da lista de símbolos.
Cada vez que um novo símbolo precisa ser definido, basta adicionar a seguinte linha:
\begin{lstlisting}[numbers=none]
\nomenclature{$f_i$}{Fugacidade do componente $i$}
\end{lstlisting}

Se o símbolo tiver unidades de medida, esta pode ser informada
com o comando \code{nomunit} para que apareça também na listagem:
\begin{lstlisting}[numbers=none]
\nomenclature{$J^V_{i}$}{Fluxo molar de $i$ no vapor\nomunit{mol/(m^2\ s)}}
\end{lstlisting}


Para que o símbolo seja categorizado como uma letra grega, adicione um \code{G} na definição,
conforme abaixo:
\begin{lstlisting}[numbers=none]
\nomenclature[G]{$\gamma_i$}{Coeficiente de atividade do componente $i$}
\end{lstlisting}

Para a definição de siglas, adicione um \code{Z} na definição,
conforme abaixo:
\begin{lstlisting}[numbers=none]
\nomenclature[Z]{EMSO}{Environment for Modeling, Simulation and Optimization}
\end{lstlisting}

De forma similar ao exemplificado acima, sobrescritos e subscritos podem
também ser definidos separadamente com o auxílio das letras \code{R} e \code{S}, respectivamente.

Todos os símbolos e siglas definidos desta forma serão automaticamente adicionados
na listagem de símbolos.

\section{Usando o pacote siunitx}

O pacote \code{siunitx} está carregado no padrão \LaTeX{} do PPGEQ.
Com ele se torna simplificada a exibição de números decimais.
Por exemplo, se utilizamos o comando \verb|\num{3.45e-4}| teremos como resultado \num{3.45e-4}.

Ainda com este pacote, a impressão de unidades de medida também é facilitada.
Por exemplo, \verb|\SI{250}{\celsius}| resulta em \SI{250}{\celsius} ou então
\verb|\si{m^2/s}| vai ser apresentado como \si{m^2/s}.

\section{Usando o pacote mhchem}

O pacote \code{mhchem} está carregado no padrão \LaTeX{} do PPGEQ.
Com o axílio deste pacote é fácil escrever equações para reações químicas.
Por exemplo, com o seguinte comando \verb|\ce{H2O + H2O -> H3O+ + OH-}| teremos:
\begin{equation}
\ce{H2O + H2O -> H3O+ + OH-}
\end{equation}

Uma fórmula no meio do texto pode ser inserida simplesmente
com \verb|\ce{H2O}|, que será formatada da seguinte forma: \ce{H2O}.

\subsection{Equações mais elaboradas}

Equações bem mais elaboradas podem ser escritas.
Uma lista de exemplos segue abaixo, consulte os fontes deste documento ou
a documentação do \code{mhchem} para mais detalhes.

\begin{equation}
\ce{2LiOH_{(s)} + CO_{2(g)} -> Li_{2}CO_{3(s)} + H_{2}O_{(g)} }
\end{equation}

\begin{equation}
\ce{CO2 + C <=> 2CO}
\end{equation}

\begin{equation}
\ce{H+ + OH- <=>> H2O}
\end{equation}

\begin{equation}
\ce{CO2 + C ->[\alpha][\beta] 2CO}
\end{equation}


\begin{equation}
\ce{SO4^2- + Ba^2+ -> BaSO4 v}
\end{equation}

